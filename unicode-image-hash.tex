% !TEX TS-program = XeLaTeX
% The next lines tell TeXShop to typeset with xelatex, and to open and save the source with Unicode encoding.

%!TEX encoding = UTF-8 Unicode
% !BIB TS-program =

\documentclass[12pt]{article}

\usepackage{uproposal}
\usepackage{textcomp}

%\usepackage{draftwatermark}
%\SetWatermarkText{DRAFT}
%\SetWatermarkScale{1}

\title{Coded Hashes of Arbitrary Images\\\normalsize \sf (Revised Preliminary Proposal)}
\author{
Steven R.~Loomis\\
\small srl@icu-project.org\\
\small (individual contribution)\\
\and
Keith Winstein\\
\small keithw@cs.stanford.edu\\
\small Stanford University\\
\and
Jennifer 8. Lee\\
\small jenny@jennifer8lee.com\\
\small Emojination\\
}

\date{2016-11-08\\\small\url{https://srl295.github.io}}


\makeindex

\begin{document}

\maketitle

\section{Introduction}
\addcontentsline{toc}{section}{Introduction}

This proposal is an update to L2/16-105\autocite{L216105}.

\begin{quote}
Emoji are pictographs (pictorial symbols) that are typically presented
in a colorful cartoon form and used inline in text. [\ldots] In
Unicode 8.0, there is a total of 1,282 emoji, which are represented
using 1,051 code points.\autocite{UTR51}
\end{quote}

Recently, there has been considerable interest in adding newly created
pictorial symbols, not found in any existing character set, to the
Unicode Standard as emoji.\footnote{\textit{E.g.,}
  \href{https://www.change.org/p/unicode-consortium-the-taco-emoji-needs-to-happen-aeb4ebc7-a323-441d-90b9-20b90c83a8c6}{Taco
    Emoji campaign}, \href{http://www.beardemoji.com/}{Beard Emoji
    campaign}, \href{http://www.dumplingemoji.org}{Dumpling Emoji
    campaign}. There are currently 79 candidate emoji that have been
  assigned tentative code points.} Advocacy groups and others request
these code points because Unicode plain text remains the dominant
interoperable interchange format for messaging. In practice, before a
new emoji can be used, a code point must be assigned and be recognized
by the sender and receiver. The stated longer-term goal for Unicode is
that implementations should support ``embedded graphics, in addition
to the emoji characters''\autocite[Section 8, ``Longer Term
  Solutions'']{UTR51}.
  
In this updated proposal, we describe a mechanism to uniquely identify
arbitrary images within a plain-text Unicode character sequence. This
will allow implementers to retrieve emoji characters as glyphs from 
% deleted: "create their own emoji"
a font without needing to
request and wait for the assignment of a code point. The basic idea is
to encode a globally-unique \emph{secure hash} of the emoji in a
Unicode character sequence. Once the implementation knows the image's hash,
it may already have the corresponding image, or may have a choice of
several mechanisms to retrieve it.

Our technique will gracefully degrade on legacy Unicode
implementations, but our proposal is limited to allowing Unicode to
\emph{uniquely identify} an arbitrary image.

% We propose to leave to
%implementers the details of how to retrieve the actual image.

  \subsection{Background} 

The demand for emoji characters has surged in the last few years as public has become enthralled by the multicolor glyphs that can appear in-line with text. Emoji have been the subject of an explosion of press pieces, numerous late-night television comedic segments, an upcoming movie from Sony Pictures Animation, and even a dedicated conference in November 2016. Never has the encoding attracted such attention.

As emoji use has proliferated, there have been increasing demands for better representation in the existing emoji set — ranging from hair variations (color and facial), to more female presence, to single-family households, to regional flags.\autocite{UTR52}

However, the current emoji approval process takes upwards of 18 months, starting from when a proposal is first introduced, to when it moves out of the Unicode Emoji Subcommittee, to the Unicode Technical Committee, to ISO and back again for final approval. This current process, overseen by a relatively small organization whose stated main mission is focused on encoding existing languages, is widely understood as a bottleneck and also incredibly resource intensive on members’ time.

To fill the demand for customizable glyphs, there has been an explosion of emoji-like images from a variety of vendors, ranging from personalized Bitmoji, to highly publicized “Kimoji”, to earnest refugee-themed emoji. These rogue “moji” are not technically emoji at all, but rather images or “stickers” that can be sent through messaging apps. In addition, vendors are using technical workarounds such as the Zero Width Joiner (ZWJ) and tagging to create emoji images that do not have to be assigned code points.

Meanwhile, the variants of single emoji created by different vendors leads to confusion in the intent of the sender and recipient.\autocite{EmojiVarying} The inconsistent imagery is at tension with Unicode’s goal of having interoperable platform-neutral communication. 
A longer term architectural solution is needed. 

\section{Proposal}

\textbf{Outline}: implementations will be able to create
``implementation-defined emoji,'' and allow their users to send them
to receivers. The sender encodes a secure hash of the emoji (which
serves as an unforgeable globally unique identifier) in a sequence of
coded characters, which we call a Coded Hash of an Arbitrary Image
(CHAI).

New emoji can then be created without (1) needing permission from any other party,
(2) getting a code point assigned from any central registry (such as Unicode or ISO 10646)
(3) claiming an unknown/unassigned ZWJ sequence or (4) waiting for the publication of Unicode, UCS,
or even updated data files from the UTC Emoji SubCommittee.  Implementers could update the repertoire 
of emoji as frequently or infrequently as they wish.

%Upon receiving the CHAI sequence, the receiving implementation
%may already know the corresponding image, or may have to request it
%either from the sender or from a third party.

\subsection{Representing the emoji}

The first step to creating an implementation-defined emoji will be to
represent the emoji in a canonical form. This format will need to be
specified, but we do not do so here. As a straw-man, we suggest a JSON
document with three keys: \texttt{content-type} (any IANA-registered MIME
Media Type), \texttt{image} (a Base64-encoded suggested rendering of the emoji,
interpreted as the format named in the \texttt{content-type} field), and
\texttt{name} (name of the emoji, which may be used in fallback
rendering for visually-impaired users). We call this representation
the ``emoji description.''

\subsection{Secure hash}

The implementation will identify the emoji by taking the SHA-256 hash
of the emoji description. Given a secure hash value, it is intended to be
intractable to find a different input that produces the same
hash value. This is known as ``second-preimage resistance'';
SHA-256 is believed to have strong second-preimage resistance.

\subsection{Code points allocated to express the secure hash}

The prior proposal proposed to allocate 65,536 ($2^{16}$) code points to represent bits
of the secure hash, resulting in a 50\% efficiency in each Unicode encoding scheme (16 bits of
the hash will consume 32 bits in memory, on disk, or on the wire)\autocite{L216105}. This current proposal has a reduced encoding ``footprint'', meaning that the coded hash will take more bits
on the wire. The general mechanism remains similar to the prior proposal.

\subsection{Encoding an implementation-defined emoji}

To code a CHAI, the sender first encodes a ``fallback'' base character
that most closely approximates the implementation-defined emoji. This
will allow rendering to degrade gracefully on receivers that do not
support CHAIs.

Next, the special tag \textit{U+E0002 TAG CODED HASH MODIFIER} is sent.

Next, the sender encodes the secure hash by appending between between
multiple TAG characters (chosing from 0-9 and a-f). This will represent a prefix of the
SHA-256 hash of the emoji description. The sender chooses how many
bits of the hash to include. 128 bits, or 32 TAG
characters, is the minimum required. The risk of
including too few bits is that the hash may no longer be globally unique,
especially in the presence of an attacker who wishes to create a different
image with the same hash prefix.



Therefore, the actual encoding is:

\texttt { <\textit{basechar}> + <\textit{U+E0002 TAG CODED HASH MODIFIER}> + <\textit{ TAG [0-9a-f] }> +<\textit{ TAG [0-9a-f] }> +<\textit{ TAG [0-9a-f] }> + \ldots <\textit{U+E007F CANCEL TAG}> }

where the \textit{basechar} represents a fallback base character, and
each TAG [0-9a-f] character (from the range U+E0030 .. U+E0039 and U+E0061
.. U+E0066) represents 4 bits of the secure hash of the emoji
description.

\subsection{Receiving and rendering a CHAI}

On receipt, the receiver determines if it already knows of an emoji
whose hash matches the prefix encoded in the CHAI characters. If so,
it displays the emoji (either using the image provided in the emoji
description, or an image known locally). Otherwise, the receiver
displays the base character.

% …while it attempts to retrieve an emoji
%description whose hash matches the encoded hash prefix.
%
%We do not specify how this retrieval will take place; we expect it to
%vary based on the messaging protocol. In some protocols, it might be
%easiest for the receiver to simply ask the sender to send the full
%emoji description. In others, the receiver might consult an online
%repository run by the implementing vendor. Alternately, the receiver
%might consult a community repository where anybody can contribute any
%emoji description.

Because the CHAI serves as an unforgeable globally unique identifier
for the emoji description, the receiver is intended to have confidence
that if it finds a matching emoji description, it can
display it per the sender's intent.

% SRL - let's leave this out ,don't want to have ambiguity.
%If the terminating \textit{U+E007F CANCEL TAG} was not encountered,
%an implementation MAY attempt to lookup the hash given the available characters

%Depending on the protocol, as the input arrives, the receiver may have
%some ambiguity about when the sequence of CHAI characters ends.  A
%receiver may choose to wait until the next non-combining character
%(signaling the end of the combining character sequence), or a
%protocol-defined end-of-message signal, before retrieving the emoji
%description.


% SRL: leaving Privacy out for now, since we are de emphasizing the arbitrary retrieval
%\subsection{Privacy}
%
%As Unicode Technical Report \#51 notes: ``There are also privacy
%aspects to implementations of embedded graphics: if the graphic itself
%is not packaged with the text, but instead is just a reference to an
%image on a server, then that server could track usage.''\autocite[Section 8, ``Longer Term
%  Solutions'']{UTR51}
%
%We agree. A similar exposure could occur if the protocol provides
%for the receiver to ask the sender to send an unknown emoji description;
%this could unwittingly expose if anybody \emph{else} has ever sent the same
%emoji to the receiver.
%
%It is not the goal of this document to address all aspects of embedded
%graphics, and we intend to leave to domain experts and implementers
%the decision on how to handle these questions. For example, a client
%might behave similarly to current practice in retrieving external
%images in email. If the receiver already knows of the identified
%emoji, it can display it, but otherwise the receiver would display the
%fallback (base) character and ask the user to push a button before
%retrieving the emoji description.

\section{Example}

\subsection{chai.json}

Example Canonical Emoji Description

\lstinputlisting[breaklines,basicstyle=\tiny\ttfamily,columns=fullflexible,upquote=true]{chai.json}

\subsection{chai.png}

Sample PNG

\includegraphics{chai.png}

\subsection{chai.txt}

Sample encoding
\lstinputlisting[language=,frame=single,basicstyle=\small]{chai-encode.txt}


%Sample decoding
%\lstinputlisting[language=,frame=single,basicstyle=\small]{chai-decode.txt}


The resulting hash matches the first 80 bits of 

\small{\texttt{37d8c5403d29ec7d6f59b02690414de77b0597735a953c2a00fbea50b5f79bb5}}

\section{Character Data}

This proposal requests a total of 1 new character to be encoded.

\subsection{Character Properties}

\begin{verbatim}
E0002;TAG CODED HASH MODIFIER;Cf;0;BN;;;;;N;;;;;
\end{verbatim}

% TBD

%\subsection{Line Breaking}
%
%\begin{verbatim}
%D0000..DFFFD;CM   # Cf[65,534] IMAGE HASH                                                                                                                                         
%EFFFC..DFFFD;CM   # Cf     [2] IMAGE HASH                                                                                                                                         
%\end{verbatim}

\clearpage

\section{Alternate structure — UTS 52}

As of this writing, there is a proposal on the UTC agenda to re-activate the UTS 52 tag mechanism \autocite{L216226}.
An alternative, which would not require any additional characters encoded, would be to make use of the UTS 52 tag mechanism.

A possible alternate structure under the UTS 52 model is given in Table 1.
The net effect is replacing the proposed new character \texttt{U+E0002 TAG CODED HASH MODIFIER} with \texttt{U+E0048 TAG CAPITAL LETTER H}.


%E0030 .. U+E0039 and U+E0061.. U+E0066

\begin{table}[]
\centering
\caption{UTS52 Hash Sequence}
%\label{my-label}
\begin{tabular}{| r | l |}
\hline
	tag\_key	&	tag-H \\
\hline
	tag\_base	&	Any character with Emoji=Yes \\
\hline
	tag\_value	&	\texttt{[\textbackslash{}x{E0030}-\textbackslash{}x{E0039}\textbackslash{}x{E0061}-\textbackslash{}x{E0066}]\+}  \\
\hline
	
\end{tabular}
\end{table}

The above example could be re-encoded as:
	
\texttt{<U+2615>}\underline{H37d8c5403d29ec7d6f59b02690414de7}\texttt{<U+E007F>}

(where the \underline{H} is tag-H  and \underline{0-9a-f} are the corresponding tag characters)


% Probably not needed - small enough.
%\clearpage
%\tableofcontents
%\printindex

%\clearpage
\addcontentsline{toc}{chapter}{References}
\printbibliography
\section*{Colophon}

Repo URL: \small\url{https://github.com/srl295/srl-unicode-proposals} 


Typeset by \LaTeX . Made with \( 100\%  \) recycled bits. All opinions belong to the authors and do not reflect the opinions
of their associated employers. Thank you to the \href{http://www.computerhistory.org}{Computer History Museum} 
for hosting the ``\href{https://www.facebook.com/events/1727012724179335/}{Emoji (and more)}'' event, where this proposal was originally penned. 
\end{document}
